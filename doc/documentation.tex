\documentclass{article}
\usepackage{url}
\usepackage{listings}
\begin{document}
\title{Documentation for Fourier Series Fitting Code}
\maketitle
\section{Introduction}
% A brief overview of the FourFit program and its purpose.
The FourFit program is designed to help researchers and data analysts fit noisy data to the Fourier series and estimate the quality of the fit with confidence. The program uses the `lsqnonlin` function from the `optim` package in Octave/MATLAB to fit a Fourier series to noisy data. All code and documentation in this project have been written by OpenAI's ChatGPT, a powerful AI language model that can assist with many aspects of the software development process.



\section{Requirements}\label{sec:requirements}
% A list of the requirements necessary to run the program, including any software packages or libraries that are needed.
The FourFit program has only been tested on Octave, and requires the following packages:
\begin{itemize}
    \item optim
    \item signal
    \item stats
    \item io
\end{itemize}

\section{Installation}
% A step-by-step guide on how to install the FourFit program.
To install FourFit, please follow these steps:

\begin{enumerate}
  \item Download and install Octave from the official website \url{https://www.gnu.org/software/octave/download}.
  \item Clone the FourFit repository from GitHub using the following command:
  \begin{verbatim}
    git clone https://github.com/CharleySchaefer/FourFit.git
  \end{verbatim}
  \item Open Octave and navigate to the directory where FourFit is located.
  \item Run the \texttt{FourFit.m} script by typing the following command in Octave:
  \begin{verbatim}
    FourFit
  \end{verbatim}
  \item The FourFit program will now run, allowing you to fit Fourier series to your data.
\end{enumerate}

\section{Usage}
% A detailed explanation of how to use the program, including examples and screenshots where appropriate.
The FourFit program is designed to fit a sum of sinusoids to a given set of data points. To use the program, follow these steps:

\begin{enumerate}
\item Make sure that all the required software packages and libraries are installed on your system (see section \ref{sec:requirements}).
\item Download the FourFit script from the project's GitHub repository and save it to a directory on your computer.
\item Load the data that you want to fit into FourFit. The example data in the script is loaded as follows:
\begin{lstlisting}[language=Octave]
x = linspace(0,2*pi,100)';
y = sin(2*x) + 0.5*cos(5*x) + 0.2*sin(10*x) + 0.1*randn(size(x));
\end{lstlisting}

Here, \texttt{x} is a vector of 100 points evenly spaced between 0 and $2\pi$, and \texttt{y} is a corresponding vector of function values with added noise. You can replace this code with your own data by loading your data into \texttt{x} and \texttt{y}.

\item Run the FourFit script in Octave by typing \texttt{FourFit} in the command window. The program will prompt you to modify the variables \texttt{n\_terms}, \texttt{x\_min}, and \texttt{x\_max} to adjust the number of terms in the sum of sinusoids and the range of the independent variable. Once you have set these variables, the program will fit the sinusoids to your data and output the results.

\item The output of the program includes a plot of the fitted function and a table of the fitted coefficients. The plot shows the original data points as well as the fitted function. The coefficients table shows the amplitude, frequency, and phase of each sinusoidal term in the fit. The interpretation of these coefficients depends on the specific problem at hand, but in general, the amplitude corresponds to the strength of the sinusoid, the frequency corresponds to its rate of oscillation, and the phase corresponds to its position relative to the origin.
\end{enumerate}

Note that the example data provided in FourFit is just a demonstration of the program's capabilities and should not be used as a real-world example without appropriate modification.

\section{Options}
% A list of the available options and their descriptions.
FourFit provides the following options:
\begin{itemize}
    \item \texttt{n\_terms}: Specifies the number of Fourier series terms to use. The default value is 3.
    \item \texttt{x\_min}: Specifies the minimum value of x to use when generating the Fourier series. The default value is 0.
    \item \texttt{x\_max}: Specifies the maximum value of x to use when generating the Fourier series. The default value is $2\pi$.
    \item \texttt{fit\_method}: Specifies the fitting method to use. Currently only the \texttt{lsqnonlin} method is implemented.
\end{itemize}

\section{Output}
% An explanation of the output generated by the program and how to interpret it.
The output of the \texttt{FourFit} program consists of the best-fit Fourier series and the corresponding plot.

The best-fit Fourier series is an array of $2N+1$ coefficients, where $N$ is the number of terms specified in the \texttt{n\_terms} variable. The coefficients are ordered as follows: the first $N$ coefficients are the amplitudes of the cosine terms, the $(N+1)$-th coefficient is the DC offset, and the last $N$ coefficients are the amplitudes of the sine terms. The values of these coefficients represent the weights of the corresponding harmonic functions in the Fourier series that best approximates the input data.

The plot shows the input data as points, and the best-fit Fourier series as a line. The plot also displays the residual, which is the difference between the input data and the best-fit Fourier series. The smaller the residual, the better the fit. The plot can be saved as a PNG image file if desired.

For example, consider the sample data provided in the \texttt{FourFit} script. After running the program with the default parameters, the following best-fit Fourier series is obtained:

\section{Troubleshooting}
% A section on common errors and how to fix them.

\section{Contributing}
% Information on how to contribute to the project, including how to submit bug reports or feature requests.

\section{License}
% A statement about the license under which the program is released.


\end{document}
