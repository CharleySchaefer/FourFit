\documentclass{article}

\begin{document}
\title{Documentation for Fourier Series Fitting Code}
\maketitle
\section{Introduction}
% A brief overview of the FourFit program and its purpose.
The FourFit program is designed to help researchers and data analysts fit noisy data to the Fourier series and estimate the quality of the fit with confidence. The program uses the `lsqnonlin` function from the `optim` package in Octave/MATLAB to fit a Fourier series to noisy data. All code and documentation in this project have been written by OpenAI's ChatGPT, a powerful AI language model that can assist with many aspects of the software development process.



\section{Requirements}
% A list of the requirements necessary to run the program, including any software packages or libraries that are needed.
Before you can use FourFit, you'll need to make sure you have the following requirements installed:

\begin{itemize}
\item Octave or MATLAB
\item The \texttt{optim} package in Octave/MATLAB
\end{itemize}

Octave is a free and open-source software similar to MATLAB, and it can be downloaded from the Octave website (\url{https://www.gnu.org/software/octave/}). MATLAB is a commercial software package that can be purchased from MathWorks (\url{https://www.mathworks.com/products/matlab.html}).

To install the \texttt{optim} package in Octave/MATLAB, follow the instructions in the Octave or MATLAB documentation.

\section{Installation}
% A step-by-step guide on how to install the FourFit program.

\section{Usage}
% A detailed explanation of how to use the program, including examples and screenshots where appropriate.

\section{Options}
% A list of the available options and their descriptions.

\section{Output}
% An explanation of the output generated by the program and how to interpret it.

\section{Troubleshooting}
% A section on common errors and how to fix them.

\section{Contributing}
% Information on how to contribute to the project, including how to submit bug reports or feature requests.

\section{License}
% A statement about the license under which the program is released.


\end{document}
